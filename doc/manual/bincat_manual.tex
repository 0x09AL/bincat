\documentclass{report}
\usepackage{algorithm2e}
\usepackage{tikz}
\usetikzlibrary{arrows, snakes, shapes, positioning, shadows, trees}
\title{User and Reference Manual of BinCAT}
\author{Sarah Zennou\\sarah.zennou@airbus.com}
\begin{document}
\maketitle
\tableofcontents
\chapter{Installation}

\section{Dependencies}
\chapter{Documentation}

can be found in html and latex format for both python and ocaml \newline
in doc/generated/[python|ocaml]/[html|latex]

\chapter{Color rules for tainting analysis}
in Bincat Tainting Vue
green : tainted
black not tainted
blue: unknown tainting
yellow: partial tainting

in IDA view:
gray: instruction has been analysed
green: at least one operand source of the instruction is tainted

rule for call:
- a call is colored if at least one of its instruction is colored
- ecx is tainted in repne scasb if at least one bit of edi is tainted. It receives the strongest tainted value of edi

noms Node dans BinCAT View
ALL, NONE

idee formation:
sur le meme code, mettre un taintage sur puis rejouer avec taintage incertain ou en backward
trouver un endroit de taint imprecise et faire clic droit taint/untaint de la destination (on ne le fait que sur des set)
\end{document}
